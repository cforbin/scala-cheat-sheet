\documentclass[10pt,landscape,a4paper]{article}
\usepackage[utf8]{inputenc}
\usepackage[ngerman]{babel}
\usepackage{tikz}
\usetikzlibrary{shapes,positioning,arrows,fit,calc,graphs,graphs.standard,trees}
\usepackage[nosf]{kpfonts}
\usepackage{multicol}
\usepackage{wrapfig}
\usepackage[top=0mm,bottom=1mm,left=0mm,right=1mm]{geometry}
\usepackage[framemethod=tikz]{mdframed}
\usepackage{microtype}
\usepackage{minted}
\usepackage{titlesec}
\usepackage{keystroke}

\definecolor{mygreen}{HTML}{4F8D4F}
\definecolor{myblue}{HTML}{3C6A6A}
\definecolor{myyellow}{HTML}{B18663}
\definecolor{myred}{HTML}{B16363}
\definecolor{sectionbg}{HTML}{D5D5D5}

\tikzstyle{every node}=[font=\footnotesize\sffamily]
\tikzstyle{sup}=[rectangle,draw,fill=myblue!50,rounded corners=.8ex]
\tikzstyle{sub}=[rectangle,draw,fill=mygreen!70,rounded corners=.8ex]
\tikzstyle{block} = [rectangle, draw, text width=2.5cm, text centered, rounded corners, minimum height=4em,fill=blue!20]
\tikzstyle{line} = [draw,thick, -latex']
\tikzstyle{cloud} = [draw, ellipse, text width=2.5cm, text centered]
\tikzstyle{edge from parent}=[<-,thick,draw]

\let\bar\overline

\def\firstcircle{(0,0) circle (1.5cm)}
\def\secondcircle{(0:2cm) circle (1.5cm)}

\colorlet{circle edge}{myblue}
\colorlet{circle area}{myblue!5}

\tikzset{filled/.style={fill=circle area, draw=circle edge, thick},
    outline/.style={draw=circle edge, thick}}

\pgfdeclarelayer{background}
\pgfsetlayers{background,main}

\everymath\expandafter{\the\everymath \color{myblue}}
\everydisplay\expandafter{\the\everydisplay \color{myblue}}

\renewcommand{\baselinestretch}{.8}
\pagestyle{empty}

\global\mdfdefinestyle{header}{%
linecolor=gray,linewidth=1pt,%
leftmargin=0mm,rightmargin=0mm,skipbelow=0mm,skipabove=0mm,
}

\newcommand{\header}{
\begin{mdframed}[style=header]
\footnotesize
The codecentric Scala Cheat Sheet
\end{mdframed}
}

\makeatletter

\newcommand{\colorsection}[1]{\colorbox{sectionbg}{\parbox{\textwidth-215\fboxsep}{\centering#1}}}
\titleformat{\section}[block]{\color{myblue}\sffamily\small\bfseries\filcenter}{}{1em}{\colorsection}
\titleformat{\subsection}[block]{\color{myblue}\sffamily\small\bfseries\filcenter}{}{1em}{}

\def\multi@column@out{%
   \ifnum\outputpenalty <-\@M
   \speci@ls \else
   \ifvoid\colbreak@box\else
     \mult@info\@ne{Re-adding forced
               break(s) for splitting}%
     \setbox\@cclv\vbox{%
        \unvbox\colbreak@box
        \penalty-\@Mv\unvbox\@cclv}%
   \fi
   \splittopskip\topskip
   \splitmaxdepth\maxdepth
   \dimen@\@colroom
   \divide\skip\footins\col@number
   \ifvoid\footins \else
      \leave@mult@footins
   \fi
   \let\ifshr@kingsaved\ifshr@king
   \ifvbox \@kludgeins
     \advance \dimen@ -\ht\@kludgeins
     \ifdim \wd\@kludgeins>\z@
        \shr@nkingtrue
     \fi
   \fi
   \process@cols\mult@gfirstbox{%
%%%%% START CHANGE
\ifnum\count@=\numexpr\mult@rightbox+2\relax
          \setbox\count@\vsplit\@cclv to \dimexpr \dimen@-1cm\relax
\setbox\count@\vbox to \dimen@{\vbox to 1cm{\header}\unvbox\count@\vss}%
\else
      \setbox\count@\vsplit\@cclv to \dimen@
\fi
%%%%% END CHANGE
            \set@keptmarks
            \setbox\count@
                 \vbox to\dimen@
                  {\unvbox\count@
                   \remove@discardable@items
                   \ifshr@nking\vfill\fi}%
           }%
   \setbox\mult@rightbox
       \vsplit\@cclv to\dimen@
   \set@keptmarks
   \setbox\mult@rightbox\vbox to\dimen@
          {\unvbox\mult@rightbox
           \remove@discardable@items
           \ifshr@nking\vfill\fi}%
   \let\ifshr@king\ifshr@kingsaved
   \ifvoid\@cclv \else
       \unvbox\@cclv
       \ifnum\outputpenalty=\@M
       \else
          \penalty\outputpenalty
       \fi
       \ifvoid\footins\else
         \PackageWarning{multicol}%
          {I moved some lines to
           the next page.\MessageBreak
           Footnotes on page
           \thepage\space might be wrong}%
       \fi
       \ifnum \c@tracingmulticols>\thr@@
                    \hrule\allowbreak \fi
   \fi
   \ifx\@empty\kept@firstmark
      \let\firstmark\kept@topmark
      \let\botmark\kept@topmark
   \else
      \let\firstmark\kept@firstmark
      \let\botmark\kept@botmark
   \fi
   \let\topmark\kept@topmark
   \mult@info\tw@
        {Use kept top mark:\MessageBreak
          \meaning\kept@topmark
         \MessageBreak
         Use kept first mark:\MessageBreak
          \meaning\kept@firstmark
        \MessageBreak
         Use kept bot mark:\MessageBreak
          \meaning\kept@botmark
        \MessageBreak
         Produce first mark:\MessageBreak
          \meaning\firstmark
        \MessageBreak
        Produce bot mark:\MessageBreak
          \meaning\botmark
         \@gobbletwo}%
   \setbox\@cclv\vbox{\unvbox\partial@page
                      \page@sofar}%
   \@makecol\@outputpage
     \global\let\kept@topmark\botmark
     \global\let\kept@firstmark\@empty
     \global\let\kept@botmark\@empty
     \mult@info\tw@
        {(Re)Init top mark:\MessageBreak
         \meaning\kept@topmark
         \@gobbletwo}%
   \global\@colroom\@colht
   \global \@mparbottom \z@
   \process@deferreds
   \@whilesw\if@fcolmade\fi{\@outputpage
      \global\@colroom\@colht
      \process@deferreds}%
   \mult@info\@ne
     {Colroom:\MessageBreak
      \the\@colht\space
              after float space removed
              = \the\@colroom \@gobble}%
    \set@mult@vsize \global
  \fi}

\makeatother
\setlength{\parindent}{0pt}

\begin{document}
\setminted{style=colorful}

\small
\begin{multicols*}{4}
  \section{Variables}
  \begin{minted}{scala}
var x      = "mutable"
val y      = "immutable"
lazy val z = "lazy"
\end{minted}

\section{Methods}
\begin{minted}{scala}
def id(arg: Char): Char = arg
def add(x: Int, y: Int): Int = x+y
def add(x: Int)(y: Int): Int = x+y
def twice[A](f: => A) = { f; f }
\end{minted}

\section{Function Types}
\begin{center}
\begin{minipage}{0.25\linewidth}
\begin{minted}{scala}
   () => A
 Unit => A
    A => B
(A,B) => C
\end{minted}
\end{minipage}
\begin{minipage}{0.25\linewidth}
  \begin{minted}{text}
    ===
    ===
    ===
    ===
  \end{minted}
\end{minipage}
\begin{minipage}{0.4\linewidth}
\begin{minted}{scala}
Function0[A]
Function1[Unit,A]
Function1[A,B]
Function2[A,B,C]
\end{minted}
\end{minipage}
\end{center}
\section{Types}
\begin{center}
\begin{tikzpicture}[auto,grow=right]
\tikzstyle{level 1}=[sibling distance=16em,level distance=3em]
\tikzstyle{level 2}=[sibling distance=4em,level distance=6em]
% Place nodes
\node [sup] (any) {Any}
child{node [sup] (anyval) {AnyVal}
    child{node [sub] (nums) {Numbers}}
    child{node [sub] (char) {Char}}
    child{node [sub,fill=myred!60] (char) {Your AnyVal}}
    child{node [sub] (bool) {Boolean}}
}
child{node [sup] (anyref) {AnyRef}
    child{node [sub] (colls) {Collections}}
    child{node [sub,fill=myred!60] (clss) {Your Class}}
    child{node [sub] (str) {String}}
};
\node[sup,xshift=1cm,right of=clss](null){Null};
\node[sup,xshift=4.5cm,right of=any](nothing){Nothing};
%% Draw edges
\path [line] (null) -- (str);
\path [line] (null) -- (colls);
\path [line] (null) -- (clss);
\path [line] (nothing) -- (null);
\path [line] (nothing) -- (bool);
\path [line] (nothing) -- (char);
\path [line] (nothing) -- (nums);
\end{tikzpicture}
\end{center}
\section{Collections}
\begin{center}
  \begin{tikzpicture}[
    level 1/.style={sibling distance=10em}]
    \node[sup]{Option[+A]}
    child{node[sub] {None[Nothing]}}
    child{node[sub]{Some[+A](value: A)}}
    ;
  \end{tikzpicture}
\end{center}
\begin{center}
  \begin{tikzpicture}[
    level 1/.style={sibling distance=10em}]
    \node[sup]{Either[+A,+B]}
    child{node[sub] {Left[+A](value: A)}}
    child{node[sub]{Right[+B](value: B)}}
    ;
  \end{tikzpicture}
\end{center}
\begin{center}
  \begin{tikzpicture}[
    level 1/.style={sibling distance=12em}]
    \node[sup]{Try[+A]}
    child{node[sub] {Failure[+A](e: Throwable)}}
    child{node[sub]{Success[+A](value: A)}}
    ;
  \end{tikzpicture}
\end{center}
\subsection{Either}
\begin{itemize}
\item getOrElse
\item fold
\item ...
\end{itemize}
\subsection{Either}
\begin{itemize}
\item getOrElse
\item fold
\item ...
\end{itemize}
\subsection{Either}
\begin{itemize}
\item getOrElse
\item fold
\item ...
\end{itemize}
\subsection{Either}
\begin{itemize}
\item getOrElse
\item fold
\item ...
\end{itemize}
\subsection{Either}
\begin{itemize}
\item getOrElse
\item fold
\item ...
\end{itemize}
\subsection{Either}
\begin{itemize}
\item getOrElse
\item fold
\item ...
\end{itemize}
\subsection{Try}
\begin{itemize}
\item getOrElse
\item fold
\item ...
\end{itemize}

\section{Futures}

\subsection{Timeouts}
\begin{itemize}
\item
  \begin{minted}{scala}
    import scala.concurrent.duration._
  \end{minted}
\item Await.result(future, 42.seconds)
\end{itemize}

\section{Implicits}
\begin{itemize}
\item implicit conversions
\item implicit arguments
\end{itemize}

\section{Type Parameters}


\section{Type Hierarchy}
Text 2

\section{IntelliJ}
\begin{tabular}{l l}
  Reformat & \Ctrl+\Alt+\keystroke{L} \\
  Extract Method & \Ctrl+\Alt+\keystroke{M} \\
  Rename & \Shift+\keystroke{F6} \\
  Expand selection & \Ctrl+\keystroke{W} \\
  Surround with & \Ctrl+\Alt+\keystroke{T}
\end{tabular}

\end{multicols*}
\end{document}